\section{Experiments}
\label{sec:experiments}

\subsection{Discovering phonotactic constraints}
\label{sec:phonotactics}

Focusing on German and Italian, that have reasonably transparent
orthographies.

How setup here differs from general one.

Method: construct pairs of letter bigrams (corresponding to phoneme
bigrams) beginning with the same letter, such that one is
phonotactically acceptable in the language and the other isn't, but
the independent unigram probability of the unacceptable bigram is
higher than that of the acceptable one. E.g., ``\emph{br}'' is
acceptable Italian sequence, ``\emph{bt}'' isn't, although
\emph{``t''} is more frequent. We re-train the CNLM on a version of
the corpus from which both bigrams have been removed. We then look at
the likelihood the model assigns to both sequences. If the model
assigns a larger probability to the correct sequence, it means that it
implicitly possesses a notion of phonological categories such as
stops and sonorants, which allows it to correctly generalize from
attested (e.g., ``\emph{tr}'') sequences to unattested ones
(\emph{``br''}).

Results, in a table with all pairs for both languages?

LSTM vs RNN

Discussion: note that generalization of model are purely
distributional, with no aid from perceptual or articulatory cues.

\subsection{Word segmentation}
\label{sec:segmentation}

English/German/Italian

Specifics of this setup, and differences from general setup

Does the model develop an implicit notion of word?

Should we use only one method here? One that is unsupervised, for
direct comparison to the Bayesian approach?

LSTM vs RNN

Report syntactic depth plot: illustrates how it is useful for
segmentation knowledge to be implicit, as the model ``knows'' about
different kinds of boundaries in a continuous manner.

Qualitative analysis: look at common over- and under-segmentation
errors in English? E.g., I could go manually through the top 30 ones,
say...

\subsection{Discovering morphological categories}
\label{sec:categories}

Focusing on German and Italian given massive morphosyntactic ambiguity
and impoverished morphology of English. Note that these are lexical
properties, probed in a model that has no explicit notion of word!

\paragraph{Word classes (nouns vs.~verbs)}

Departures from general setup

Procedure as described in the quip.

Baselines: autoencoder, word-based NLM embeddings, also LSTM vs
RNN. Outperforming autoencoder shows that model has learned categories
based on broader distributional evidence, not just typical strings
cueing nouns and verbs.

\paragraph{Number}

Does the hidden state of the CNLM store an abstract notion of
number. German nouns can be binned in different classes depending on
the morpheme or morphological process they use to form the plural. We
train a number classifier on a subset of these classes, test on the
others: if model generalizes correctly, it means that it knows about
number independently of its surface expression.

Specifics of data-set construction and classifier training.

Baselines/comparisons: LSTM vs RNN, autoencoder, word-based NLM.

Control follow-up: singular nouns with plural ending.

\subsection{Capturing syntactic dependencies}
\label{sec:dependencies}

Despite not having pre-defined information about words and morphemes,
is the model able to capture long-distance syntactic dependencies? NB:
actually for a CNLM even \emph{``\textbf{la} bell\textbf{a}''} is long
distance!

To be continued\ldots

\subsection{Lexical semantic similarity}
\label{sec:similarity}

In English, because that's where we have resources available.

Correlation with one or more word similarity sets.

Comparison to word-based NLM (rather than word2vec or such, which is
specifically tuned for semantics).

