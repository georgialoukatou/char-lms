
\subsubsection{Italian} We focus on a  subset of Italian morphology where gender and number are explicitly and systematically
encoded while allowing for tightly controlled comparison of
same-length strings, limited to stimuli unseen in the training corpus.
\begin{inparaenum}[i)]
\item article-noun gender agreement with material in the middle,
\item article-adjective gender agreement, with an adverb in the middle,
\item article-adjective number agreement, with an adverb in the middle.
\end{inparaenum}

\paragraph{Article-noun gender agreement}
%(1) eadj-aonoun:

Similar to German, Italian articles agree with the noun in gender; however, while gender is often unpredictable in German, Italian has a productive system of forming masculine and feminine versions of nouns that differ in the final vowel (-\emph{o} for masculine, -\emph{a} for feminine).
We construct stimuli of the form:
\begin{enumerate}[label={(\arabic*)}]
	\item 
		\begin{tabular}[t]{lllllll}
	a. & \{\underline{il}, la\} & congeniale & candidato \\
   &  the & congenial & candidate (m.) \\
	& \multicolumn{4}{l}{`The congenial male candidate.'} \\
	b. & \{il, \underline{la}\} & congeniale & candidata \\
    &the & congenial & candidate (f.) \\
	& \multicolumn{4}{l}{`The congenial female candidate.'} \\
\end{tabular}
\end{enumerate}

Note that the intervening adjective, ending in -\emph{e}, does not reveal the noun's gender, increasing the distance across which gender information has to be transported.

We constructed these stimuli such that none of the adjective-noun pairs appear in the training data, and all single words appear at least 100 times.
Further, the nouns in -\emph{a} and  -\emph{o} have reasonably balanced frequency (neither form is twice more frequent than the other), or they are both frequent (appear at least 500 times)
As the prenominal adjectives are somewhat marked in Italian, we  we considered only -e adjectives that occur at least with 10 different nouns in the prenominal position.

Results are shown in the first line in Table~\ref{tab:ital-agr-results}.
The word LSTM shows the strongest performance, closely followed by the LSTM CNLM.
Even the RNN CNLM performs strongly above chance.

\paragraph{Article-adjective gender agreement}
We then created an experiment where an adverb intervened between an article and an adjective, both of which were marked for gender:
\begin{enumerate}[label={(\arabic*)}]
	\item 
		\begin{tabular}[t]{lllllll}
	a. & il & meno & \{ \underline{alieno}, aliena \} \\
   &  the & less & alien one  \\
	b. & la & meno & \{ alieno, \underline{aliena} \} \\
    &the & less & alien one \\
\end{tabular}
\end{enumerate}
where we used the adverbs \emph{pi{\`u}} `more', \emph{meno} `less', \emph{tanto} `so much'.
We considered only adjectives that occurred 1K times in the training corpus, as this is the most common kind of adjective in Italian.
We excluded all cases in which the adverb-adjective combination occurred in the training corpus, in either feminine or masculine form.


% /checkpoint/mbaroni/char-rnn-exchange/candidate_adv_aoadj_testset.txt

Results are shown in the second line in Table~\ref{tab:ital-agr-results}; all three models perform almost perfectly.

\begin{table}[t]
  \begin{center}
    \begin{tabular}{l|ll|ll|ll}
	    & \multicolumn{4}{c|}{CNLM} & \multicolumn{2}{c}{\multirow{2}{*}{WordNLM}}\\
	    &\multicolumn{2}{c|}{\emph{LSTM}}&\multicolumn{2}{c|}{\emph{RNN}} &\\ \hline
% eadj-aonoun
	    Noun Gender & 97&90  & 84&73 & 99&96 \\
%      adv-aoadj
	    Adj. Gender & 99&100 & 100&97 & 98&100 \\
% adv-aeadj
	    Adj. Number & 99&99 & 99&70 & 100&100 \\
    \end{tabular}
  \end{center}
	\caption{\label{tab:ital-agr-results} Results for morphosyntactic tests in Italian. For each model and test, we show accuracy on the two types of stimuli (masculine/feminine for gender, singular/plural for number).}
\end{table}

\paragraph{Article-adjective number agreement}
We then constructed a version of the last test that probed number agreement instead of gender agreement; number marking is similarly very systematic in Italian.
Stimuli had the form
\begin{enumerate}[label={(\arabic*)}]
	\item 
\begin{tabular}[t]{lllllll}
	a. & la & meno & \{ \underline{aliena}, aliene \} \\
   &  the & less & alien one(s)  \\
	b. & le & meno & \{ aliena, \underline{aliene} \} \\
    &the & less & alien one(s) \\
\end{tabular}
\end{enumerate}
% /checkpoint/mbaroni/char-rnn-exchange/candidate_adv_aeadj_testset.txt
Selection of stimuli was  as before, but  we took a 500-occurrences threshold, as feminine plurals are less common.
Further, we manually removed adjectives that did not combine well semantically with the adverbs under consideration (\emph{pi{\`u}, meno, tanto}).

Results are shown in the third line in Table~\ref{tab:ital-agr-results}; the LSTMs perform almost perfectly, while the RNN still performs strongly above chance.


