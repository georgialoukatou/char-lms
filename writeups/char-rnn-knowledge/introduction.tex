\section{Introduction}
\label{sec:introduction}

Recurrent neural networks \cite[RNNs,][]{Elman:1990}, in particular
in their Long-Short-Term-Memory variant
\cite[LSTMs,][]{Hochreiter:Schmidhuber:1997}, are the current
workhorse of natural language processing. RNNs, often
pre-trained on the simple \emph{language modeling} objective of
predicting the next symbol in natural text, are a crucial
component of state-of-the-art architectures for machine
translation, natural language inference and text categorization
\cite{Goldberg:2017}.

RNNs are very general devices for sequence processing, assuming little
prior bias. Moreover, the simple prediction task they are trained on
in language modeling seems well-aligned with the core role prediction
plays in cognition \cite[e.g.,][]{Bar:2007,Clark:2016}. RNNs have thus
long attracted researchers
interested in language acquisition and processing. Their recent successes in
realistic large-scale tasks has strongly rekindled this interest
\cite[see, e.g.,][and references there]{Frank:etal:2013,Lau:etal:2017,Kirov:Cotterell:2018,McCoy:etal:2018,Pater:2018}.

The standard pre-processing pipeline of modern RNNs assumes that the
input has been tokenized into word units that are pre-stored in the
RNN vocabulary. This is a reasonable practical approach, but it makes
simulations less interesting from a linguistic point of view. First,
discovering words is one of the major challenges a learner faces, and
by pre-encoding them in the RNN we are facilitating its task in an
unnatural way (not even the staunchest nativists would take specific
word dictionaries to be part of our genetic code). Second, assuming a
unique tokenization into a finite number of discrete word units is in
any case problematic. The very notion of what counts as a word in
languages with a rich morphology is far from clear
\cite[e.g.,][]{Bickel:Zuniga:2017}, and, universally, mental lexicons
are probably organized into a not-necessarily-consistent hierarchy of
units at different levels: morphemes, words, compounds, constructions,
etc.~\cite[e.g.,][]{Goldberg:2005}.

Motivated by these considerations, we present here an extensive study
of RNNs trained on language modeling at the character level, or
\emph{character-level neural language models}
\cite[CNLMs,][]{Mikolov:etal:2011,Sutskever:etal:2011,Graves:2014}. Moreover,
we trained the RNNs on input where whitespace has been removed, so
that, like children learning a language, they don't have access to
explicit cues to wordhood.\footnote{We do not erase punctuation marks,
  reasoning that they have a similar function to prosodic cues in
  spoken language.} This setup is almost as \emph{tabula rasa} as it
goes. By using unsegmented orthographic input (and assuming that, in
the alphabetic writing systems we work with, there is a reasonable
correspondence between letters and phonetic segments), we are only
postulating that the learner figured out how to segment the continuous
speech stream into phonological units, an ability children
already possess few months after birth
\cite[e.g.,][]{Maye:etal:2002,Kuhl:2004}.

After training the networks on the unsupervised character-level
language modeling task, we probe them with phonological, lexical,
morphological, syntactic and semantic tests in English, German and
Italian. Our results show that near-\emph{tabula-rasa} CNLMs acquire
an impressive spectrum of linguistic knowledge at various levels.
This in turn suggests that, given abundant input (large Wikipedia
dumps), a learning device whose only prior architectural bias consists
in the LSTM memory cell implicitly acquires a variety of linguistic
rules that one would intuitively expect to require much more prior
knowledge.\footnote{Upon publication, we will make our input data, test sets and pre-trained model available.}
