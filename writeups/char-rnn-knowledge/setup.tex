\section{Experimental setup}
\label{sec:setup}

We downloaded full Wikipedia dumps for English, German, and Italian, and extracted plain text with WikiExtractor.\footnote{https://github.com/attardi/wikiextractor}
For each language, we randomly extracted testing and validation sections consisting of 50,000 paragraphs each, and used the remainder as training sets.
The training sets contained 16M (German), 9M (Italian), and 41M (English) paragraphs, corresponding to 819M (German), 463M (Italian), and 2333M (English) words.
Order of paragraphs was shuffled for training; we did not attempt to split by sentences.
For word segmentation and word-based language models, we tokenized and tagged all corpora using TreeTagger.\footnote{http://www.cis.uni-muenchen.de/~schmid/tools/TreeTagger/}

For the vocabulary, we extracted the most frequent characters from each corpus, setting thresholds so as to ensure that all characters representing phonemes of the languages were included.
The vocabulary of the word-based models included the most frequent 50,000 words per corpus.


For LSTM and RNN CNLMs, and for the word-level model, we each specified a range of hyperparameter settings, and applied random search to find models that performed best on the validation set.
%We took X samples for the LSTM CNLM per language, Y for the RNN CNLM, and Z for the word-level LSTM.
We terminated training after 72 hours; none of the models had overfitted, as measured by performance on the validation set.

Performance on the test partitions is shown in Table~\ref{tab:lm-results}.

Direct comparison with the state-of-the-art in character-based language modeling is hindered by the fact that we train on text without whitespace.
Consequently, our BPS do not look as good as other character LM work, but this is expected given that we do not provide whitespace to the model: Whitespace is both relatively easy to predict, and it makes predicting other characters easier.

%Describe source corpus for each language, and how it was
%pre-processed. Training/validation/test partitioning.

%Hyperparameter search. LSTM vs RNN. Word-level model (was it also
%optimized?)

%Report results in terms of BPS on test partition, with ballpark
%estimate of where we are in terms of the state of the art.

\begin{table}[t]
  \begin{center}
    \begin{tabular}{l|l|l|l|l}
      \multicolumn{1}{c}{}&\emph{LSTM}&\emph{RNN}&\emph{Word LSTM}\\
      \hline
	    English & 1.12 / 1.62 & 1.44 / 2.08 & TODO  \\
	    German &  1.05 / 1.51 & 1.27 / 1.83 & 3.63 / 37.96   \\
	    Italian & 1.02 / 1.47 & 1.37 / 1.97 & 3.85 / 42.02  \\
    \end{tabular}
  \end{center}
  \caption{\label{tab:lm-results} For CNLMs, we report cross-entropy and bits-per-character. For word-based models, we report cross-entropy and perplexity.}
\end{table}




